%%%%%%%%%%%%%%%%%%%%%%%%% DO NOT CHANGE %%%%%%%%%%%%%%%%%%%%%%%%%%%
\documentclass[12pt]{article}
%%%%%%%%%%%%%%%%%%%%% END OF DO NOT CHANGE %%%%%%%%%%%%%%%%%%%%%%%

\usepackage[utf8]{inputenc}
\usepackage{setspace}
\usepackage{cite}




\setlength{\marginparwidth}{0pt}
\setlength{\marginparsep}{0pt} 
\setlength{\evensidemargin}{0.25in}  

\setlength{\oddsidemargin}{0.25in} 
\setlength{\textwidth}{6.375in}

%%%%%%%%%%%%%%%%%%%%% END OF DO NOT CHANGE %%%%%%%%%%%%%%%%%%%%%%%
\title{Research Question}





\author{Andy Mai, Md Iztiba, Phu Anh Pham, Bo Wei Yao, TJ LeBlanc}
\date{\today}


%%%%%%%%%%%%%%%%%%%%%%%%% DO NOT CHANGE %%%%%%%%%%%%%%%%%%%%%%%%%%%
\linespread{1.5}
%%%%%%%%%%%%%%%%%%%%% END OF DO NOT CHANGE %%%%%%%%%%%%%%%%%%%%%%%

\begin{document}
\maketitle

\section*{Research Proposal}
How does remote work affect employee productivity?

\section*{Hypotheses}
Null hypothesis: remote work has no effect on employee productivity. \\
Alternative hypothesis: remote work has either a positive or negative effect on employee productivity. 

\section*{Variables}
There are a number of variables that we are considering into measuring employee productivity: 

\subsection*{Major ones}
\begin{itemize}
  \item Self leadership 
  \item Mental health 
  \item Work environment 
  \item Time
\end{itemize}

\subsection*{Minor ones}
\begin{itemize}
  \item Training and career development 
  \item Physical health 
  \item Pay
  \item Tooling (communication, work essential tools)
  \item Process and Support (help desk, manager support)
\end{itemize}


\section*{Literature Review}

\subsection*{Health}

As the world comes out of the pandemic, we can learn more about the potential health benefits and negatives of remote work. Aside from the indirect benefits of remote work for mental health and stress from increased control of their time, there are also numerous health benefits as well \cite{doi:10.1177/1529100615593273}. Some of these include a significant reduction in work related stress and exhaustion as a direct result of having that control. There is also evidence to show the lack of travel due to remote work is actually a health benefit as well, as one study has shown that there is a negative correlation between those who commute and physical activity compared to those that do not\cite{HOEHNER2012571}. Another area of consideration has been the potential time saved from commuting and what that time could potentially be used for in terms of improving ones health. Studies have shown that people who do not commute to the office tend to eat less fast food and increase time spent at a gym\cite{allen2008workplace}. 

However there are negative health considerations associated with the prevalence of remote work. The ergonomics of the working space has a major impact on health, which offices have standards and controls for but are not always present for remote work environments. Things to consider include back support, monitor height, arm rests and the location of keyboards and mice which can all contribute to detrimental work from home health and can increase risk of injury\cite{ellison2012ergonomics}. 

\subsection*{Time}

As workers may minimize commuting time by at least 1.5 hours from remote working, they can leave work early and have more time for their personal life \cite{george2022}. In addition, people experience less stress and are more productive as they can use the non-commuting time to sleep more and feel better in the morning \cite{george2022}.

Other studies also suggest that job satisfaction increases when remote employment allows employees more flexibility, autonomy, and time to satisfy their job and their time for family and life \cite{natasha2016}. For instance, people can eat dinner with their families soon after their shift ends instead of spending more than 30 minutes traveling back home and feeling irritated by the traffic congestion. Having high-quality time with family is vital to one’s emotions and mental health. And obviously, employees do more excellent quality work when they are in a pleasant mood and healthy mentality \cite{natasha2016}.

\subsection*{Work environment}

Working in a home environment can provide significant benefits over working in an office. One commonly seen problem with working in an office environment are distractions resulting from needing to constantly interact with co-workers. This is made much worse in offices that are open-plan, where noise and a lack of privacy also become very apparent concerns.\cite{Kim2013}. Working from home helps solve these issues since employees would not need to deal with co-workers in person while also not being exposed to the generally chaotic nature of office environments. Employees are also given more flexibility to interact with their families due to not having to spend a large portion of their day away from home, which can contribute to a better work-life balance that improves both mental and physical health. \cite{Xiao2021}. Additionally, it’s essential for a work environment to have good lighting, ergonomics, air quality, acceptable temperature and humidity, and a low level of noise to ensure that a worker is satisfied and able to work at peak efficiency \cite{Xiao2021}. One of the benefits of working from home is that an employee has full control over these factors since it’s their own home. This gives them the ability to customize aspects of their workspace as they wish, which can give them a great deal of satisfaction and comfort in comparison to being confined to a small cubicle \cite{Xiao2021}.

\subsection*{Training and career development}

Events like COVID-19, followed by mandatory virtual or remote work settings, can have an impact on people's emotional and cognitive reactions, as well as their learning capacities and career development. COVID-19 has undoubtedly altered how Human resources professionals and executives prepare people and organizations for change during uncertain times as well as how people react to the change. Emotional control affects how people process information and develop opinions, which can have significant effects on how they prepare for and make decisions about their careers \cite{Restubog2020}. To reduce the negative effects of emotions, one must actively and deliberately look for ways to manage them by participating in emotionally uplifting activities.

Constant professional growth through virtual mentoring can also enable peak performance, regardless of one's career level. The chance for communication and professional growth are two benefits of virtual mentoring. It is organizations that will ultimately benefit from employees’ career development \cite{Yarberry2021}.  For more mutually beneficial outcomes, human resource development professionals are tasked to shift development initiatives to a virtual/remote environment in response to the new workplace normal.

Remote employees frequently experience social isolation since they do not have many face-to-face meetings and their communications with coworkers are irregular and limited \cite{Park2021}. They feel removed from decision-making processes and less connected to their organizations \cite{Virick2010}.Furthermore, in the context of distant e-work, the "out of sight, out of mind" approach might weaken the importance of personal connections. This might result in a person's career stagnation and professional growth, as well as limited access to social support systems including informal learning and mentorship \cite{Smith2018}.

\section*{Experimental Design}

\subsection*{Methodology}
We have decided to proceed with a quantitative approach for this experiment. We believe it is much easier to model and analyze patterns based on quantitative data rather than qualitative data, providing a much clearer correlation between groups of people and categories of measurements for employee productivity.

\subsection*{Data grouping}

That data collected will be grouped and aggregated with a number of demographic information that we collect from candidates. The following variables are considered:

\begin{itemize}
  \item Age group 
  \item Marital status
  \item Kids
  \item Approximate geographical location
  \item Sex
  \item Ethnicity
  \item Career field
\end{itemize}
These will be used for aggregation and grouping during the analysis, discussion and reporting phase. For example, we can use the data to answer questions such as the following:
\begin{itemize}
  \item How does physical health’s impact on employee productivity vary between different age groups?
  \item How does mental health’s impact on employee productivity vary between different races?
\end{itemize}


\subsection*{Data collection}

A questionnaire contains questions with regards to remote work and employee productivity will be sent out on various websites. The questionnaire will contain questions covering all categories of measurements mentioned in the Variables section above. However, the categories do not weigh the same. We have assigned different weightings for the categories based on research as well as personal experience. \\ \\
All questions will be asked in statements, followed by the following 5 options: 
\begin{itemize}
  \item Strongly disagree
  \item Somewhat disagree
  \item Neutral
  \item Somewhat agree
  \item Strongly agree
\end{itemize}
The candidate answering the question may only select 1 out of the above 5 options for each question. 

\subsection*{Categorical weighting}
\begin{itemize}
  \item Self leadership 10
  \item Mental health 10
  \item Work environment 9
  \item Time 8
  \item Training and career development 4
  \item Physical health 5
  \item Pay 3
  \item Tooling (communication, work essential tools) 4 
  \item Process and Support (help desk, manager support) 3
\end{itemize}

\subsection*{Question values}
Values 1 to 5 are assigned on scale between strongly disagree to strongly agree. However, the values are dependent on whether the question is a positive reinforcement to employee productivity. \\
In a positive reinforcement scenario, such as: 
\begin{itemize}
  \item I feel less stressed when I’m working from home. 
  \item I am more productive when I’m working from home. 
\end{itemize}
The option of strongly agree would be assigned a value of 5, while the strongly disagree option would be assigned a value of 1. \\
However, in a negative reinforcement scenario, such as: 
\begin{itemize}
  \item I tend to burn out from working remotely.
  \item I miss face to face contact with colleagues.
\end{itemize}
The vice versa is true. Strongly agree would be assigned a value of 1, while the strongly disagree option would be assigned a value of 5.

\subsection*{Assumptions}
There are several assumptions that we need to make in order to measure employee productivity from this experiment successfully.

\begin{itemize}
  \item Better trained / more qualified for the job = higher productivity
  \item Having better health (mental, physical) = higher productivity
  \item Higher pay = higher productivity
  \item Better tooling = higher productivity
  \item Better support= higher productivity
  \item Negative or disruptive environment = lower productivity
  \item More or complicated processes = lower productivity
\end{itemize}

\subsection*{Questionnaire}
Please see Appendix A for the complete questionnaire.

\bibliography{research_proposal}
%%%%%%%%%%%%%%%%%%%%%%%%% DO NOT CHANGE %%%%%%%%%%%%%%%%%%%%%%%%%%%

\bibliographystyle{ieeetr}
%%%%%%%%%%%%%%%%%%%%% END OF DO NOT CHANGE %%%%%%%%%%%%%%%%%%%%%%%

\appendix
\section*{Appendix A: Experiment questionnaire}

\subsection*{Example question}
I feel that working remotely has saved me time from commuting, allowing me to be more prepared for work in the morning. 
\begin{itemize}
  \item Strongly disagree
  \item Somewhat disagree
  \item Neutral
  \item Somewhat agree
  \item Strongly agree
\end{itemize}
All categorical questions will have the above answering format.
\subsection*{Categorical question}
\subsection*{Self leadership}

I have a hard time motivating myself when working remotely. \\
I feel that my creativity is reduced when not interacting with people. \\
I feel less engaged at work when working remotely. \\
I am my own leader when I’m working from home. \\
I’m in control of my schedule and pace when I'm working from home. \\
I’m capable of spending less hours accomplishing the same work as before. \\
I am more productive when I’m working from home.  \\
My job allows me to make my own decisions about how to schedule my work.

\subsection*{Mental health}

I feel less stressed when I’m working from home. \\
I feel more socially isolated when I’m working from home. \\
I crave interactions with real people when I’m working from home. \\
I have a hard time separating work from personal life. \\
I feel like I'm at work all the time.  \\
I tend to burn out from working remotely. \\
I feel isolated despite the usage of digital technologies for communication. \\
I miss face to face contact with colleagues. \\
I feel exhausted when working from home. \\
I suffer from anxiety or depression after spending prolonged periods at home. \\
I feel lonely when working from home. \\
I have increased conflicts with family members when working from home. 

\subsection*{Work environment}

I enjoy having the flexibility of working from home.  \\
I have a better work-life balance because of working from home. \\
My family (partner, kids, parents) often interfere with my work routine when I’m WFH. \\
I’m frequently distracted when working from home.  \\
Having remote work has given me plenty of room to focus on deliverable items.  \\
I am more organized when I’m working from home.  \\
I experience family work conflict when working from home. \\
I am able to create an isolated space for work at home.

\subsection*{Time} 

I feel that working remotely has saved me time from not having to commute. \\
I feel that working remotely has saved me time from not having to get up early. \\
I feel that working remotely has saved me time from not having to get home late. \\
I’m able to work on chores during downtimes at work.  \\
I have more time meeting with friends or going to the gym.

\subsection*{Training and career development}

I feel that remote work has reduced my chance of being promoted. \\
I feel that my company is offering less training programs because of remote work. \\
I am capable of working on personal development and growth when working remotely.


\subsection*{Physical health}

I feel that my physical condition has decreased as a result of not going out to work everyday.  \\
I have more time to spend on physical activities as a result of remote work. \\
My diet improves when I’m working from home.  \\
I spend less time grooming myself when I’m working from home.
My workspace is ergonomic. 

\subsection*{Pay}

My company has cut my pay as a result of remote work. \\
I’m forced to transition to part time as a result of remote work. \\
I believe my raise was reduced as a result of remote work. \\
I believe my bonus was cut as a result of remote work. 

\subsection*{Tooling} 

I had a difficult time arranging together my work station for remote work.  \\
I had a difficult time installing appropriate communication tools for my remote work.  \\
The tools  provided were good enough for my work purposes. 

\subsection*{Process and Support}

I received plenty of support from my supervisor during my remote work. \\
I received plenty of support from my coworkers during my remote work. \\
The IT support was easy to reach during my remote work. \\
The IT support was able to solve my problems during my remote work.


\subsection*{Candidate demographics questions}

What age group do you belong to? \\
18-25, 26-35, 36-45, 46-55, 56+, others \\ \\
What is your marital status? \\
Single, married, others \\ \\
Do you have any kids in the household? \\
Yes, no \\ \\
What continent do you live on? \\
Europe, North America, South America, Asian, Africa, others \\ \\
What is your sex? \\
M, F, others \\ \\
What do you identify as? \\
Caucasian, African, Asian, others \\ \\
What is your career field? \\
IT, customer service, accounting, marketing, operations, finance, others 


\end{document}
